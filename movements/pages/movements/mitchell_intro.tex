\subsection{Mitchell - Einleitung}\label{mitchell}

\noindent
Sowohl das \textit{Mitchell-Movement} selbst, als auch all seine nahen Verwandten teilt alle an dem Turnier teilnehmenden \pas in zwei getrennte Gruppen auf. In der einen Gruppe befinden sich alle \textit{Paare}, die sich im Verlaufe des Turniers durchgängig und ausschließlich auf \textit{Nord-Süd} sitzen und in der anderen Gruppe analog dazu all jene \textit{Paare}, die sich auf \textit{Ost-West} befinden werden.\\[.1cm]
Im Verlaufe des Turniers bleiben alle \ns \pas an ihren Tischen sitzen, wohingegen sich die \ew \pas nach jeder Runde einen \ti nach unten bewegen. Somit spielen niemals \pas innerhalb einer Gruppe gegeneinander.\\[.1cm]
Es ist entsprechend naheliegend, die \ns sowie \ew Gruppe getrennt voneinander zu betrachten und folglich zwei separate Endergebnisse zu ermitteln. Es ist möglich ein gemeinsames Ergebnis zu berechnen, was jedoch oftmals zu einem verzerrten Ergebnis führt.
