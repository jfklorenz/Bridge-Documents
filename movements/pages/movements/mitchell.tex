\subsection{Mitchell - Vollständig}

\noindent
In einem \textit{vollständigen Mitchell Movement} ist die Anzahl an \rus sowohl äquivalent zu der Anzahl der \ti als auch der Anazhl der \bgs innerhalb eines Turniers.\\
Folglich hat am Ende des Turniers jedes \ns \pa gegen jedes \ew \pa gespielt. Außerdem ist gewährleistet, dass jedes \pa alle \bos des Turniers gespielt hat.\\[.1cm]
Des weiteren funktioniert ein \textit{vollständiges Mitchell Movement} ausschließlich mit einer \odds Anzahl an \textit{Tischen}.\\
Möchte man jedoch mit einer \evens Anzahl an \textit{Tischen} unbedingt ein \textit{Mitchell Movements} wählen, so bieten sich verschiedene Alternativen, wie zum Beispiel das \textit{Skip-Mitchell Movements} an.\\[.2cm]

\noindent
In der folgenden Tabelle ist ein Beispiel eines \bm für ein \textit{Mitchell}-Turnier mit $5$ \textit{Tischen}, 10 \textit{Paaren}, sowie $5$ \bgs abgebildet.\\
Die $5$ \ns \pas bleiben stets an dem ihrer \textit{Paarnummer} entsprechenden \ti sitzen. In der Tabelle ist für jede \ru und jeden \ti das zugehörige \ew \pa sowie die entsprechende \bg aufgelistet.

\begin{center}
  \begin{tabular}{|c||c|c|c|c|c|}
    \hline
    \multicolumn{6}{|c||}{\ccb \textbf{Beispiel - Mitchell}}\\
    \hline
    \multicolumn{1}{|c|}{\cca \textbf{Tische}}&
    \multicolumn{1}{c|}{\cca \textbf{Runde 1}}&
    \multicolumn{1}{c|}{\cca \textbf{Runde 2}}&
    \multicolumn{1}{c|}{\cca \textbf{Runde 3}}&
    \multicolumn{1}{c|}{\cca \textbf{Runde 4}}&
    \multicolumn{1}{c|}{\cca \textbf{Runde 5}}\\
    \hline\hline
    $1$ & EW $1$ BG $A$ & EW $5$ BG $B$ & EW $4$ BG $C$ & EW $3$ BG $D$ & EW $2$ BG $E$\\
    \hline
    $2$ & EW $2$ BG $B$ & EW $1$ BG $C$ & EW $5$ BG $D$ & EW $4$ BG $E$ & EW $3$ BG $A$\\
    \hline
    $3$ & EW $3$ BG $C$ & EW $2$ BG $D$ & EW $1$ BG $E$ & EW $5$ BG $A$ & EW $4$ BG $B$\\
    \hline
    $4$ & EW $4$ BG $D$ & EW $3$ BG $E$ & EW $2$ BG $A$ & EW $1$ BG $B$ & EW $5$ BG $C$\\
    \hline
    $5$ & EW $5$ BG $E$ & EW $4$ BG $A$ & EW $3$ BG $B$ & EW $2$ BG $C$ & EW $1$ BG $D$\\
    \hline
  \end{tabular}
\end{center}
