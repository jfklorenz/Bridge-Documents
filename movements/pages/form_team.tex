\subsection{Team Movement}\label{team movement}

\noindent
Die Turnierform \textit{Teamturnier} beinhaltet alle Turniere, bei denen ein Spieler
gemeinsam mit einem Partner gemeinsam als \te antritt. In diesem Fall bezieht sich
auch das Ergebnis des Turniers nur auf die Leistung des ganzen \textit{Teams}.
Die Leistung der \pas innerhalb eines \te spiegelt sich in der sogenannten \bw wieder
und ist im entsprechenden Kapitel \ref{butlerwertung} eingesehen werden.\\[.2cm]

\noindent
Für \te-Turniere existiert kein echtes \bms wie für \pa-Turniere. Stattdessen wird
jeder Kampf zwischen zwei einzelnen \tes als separates Turnier behandelt. Anschließend
werden die Ergebnisse aller \te-Kämpfe in einer entsprechenden \vp-Tabelle
zusammengefasst.\\[.2cm]

\noindent
Bei \te-Kämpfen wird die \scy-Differenz beider Tische bestimmt und in \imps umgerechnet.
Das Resultat in \imps wird anschließend in \vps skaliert, was schließlich dem
Endergebnis entspricht.
