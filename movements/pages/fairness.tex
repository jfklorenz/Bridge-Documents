\section{Fairness von Movements}\label{fairness}

\noindent
Bridge \bms sind darauf ausgelegt, für alle beteiligten Spieler ein möglichst
\textit{faires} Spiel zu bieten. Ein \bms in dem jedes \pa gegen jedes andere
\pa genau einmal spielt ist allgemein äußerst fair. In anderen Fällen kann es somit zu
Unausgeglichenheiten kommen. Ein Paar könnte gegen mehr \textit{starke}, oder
analog weniger \textit{schwache} Gegner spielen, als andere Paare.
Insbesondere \textit{Howell Movements} sind ein bis heute mathematisch offenes Thema. TODO: CITE\\[.2cm]

\noindent
Die in \bb generierten \bms sind entsprechend ihrer Richtlinen stets so fair wie möglich.\\[.1cm]
Für weitere Ausblicke in diese Thematik wird hier auf die öffentlich einsehbare Fachliteratur verwiesen:

\begin{center}
  \begin{tabular}{|l|}
    \hline
    \multicolumn{1}{|c|}{\ccb \textbf{Fachliteratur - Fairness}}\\
    \hline
    \multicolumn{1}{|c|}{\cca \textbf{Literatur/Link}}\\
    \hline\hline
    \href{https://en.wikipedia.org/wiki/Duplicate_bridge_movements#Fairness_of_Bridge_Movements}{Wikipedia}\\
    \hdashline
    \href{https://books.google.de/books/about/Tournament_and_Duplicate_Bridge.html?id=UCPwAAAAMAAJ&redir_esc=y}{Tournament and Duplicate Bridge: A Complete Textbook ...}\\
    \hline
  \end{tabular}
\end{center}
